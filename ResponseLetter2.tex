\documentclass[12pt]{extarticle}
\usepackage{geometry}
\geometry{
a4paper,
total={170mm,257mm},
left=20mm,
top=20mm,
headheight=12pt
}

\usepackage{xcolor}
\usepackage{amssymb,amsmath,amsthm}
\usepackage{natbib}
\usepackage{commath}
\usepackage{url}
\usepackage{csquotes}

\renewcommand{\d}{{\rm d}}
\newcommand{\e}{{\rm e}}

% Title page
\title{
	Response to reviewers\\ Evolutionary rescue by aneuploidy in tumors exposed to anti-cancer drugs 
}


\begin{document}
\maketitle

%%%%%%%%%%%%%%%%%%%%%%%%%%%%%%%%
\textbf{Reviewer $\#$1}

Minor Edits:

1. Review all Figure titles (main/supplement) and turn the Figure titles into impact statements of the main result, like you did for Figures 3, 5, and 6, which give your reader an immediate connection to the purpose of the figure and the take-away. Change Figure 3 title to ``\ldots evolutionary rescue of tumors under drug treatment \ldots''. In Table 1, ``\ldots 0.01, instead \dots''. In Appendix A, ``\ldots (Kendall, 1948). This allows\ldots''. In Figure 5, ``Standing genetic variation will drive drug resistance when the sensitive cell population is rapidly declining.'' In Figure 5, ``\ldots $N_a^*$ is a function\ldots''

\textcolor{blue}{We have changed the title of Figure 4 and added a title to Figure S5. Change S18}

2. Please do a search for ``very'' across the whole manuscript and remove. The world holds no meaning here. If there is a need to describe an overwhelming amount of anything, then quantify it. 

\textcolor{blue}{We have made these changes. Change S1}

3. There is a tendency in the manuscript to use some kind of connection word to start sentences, and to do it back-to-back. For example, in Lines 22-25 in the Abstract, ``Additionally\ldots Finally,'' and lines 51-54 in the Introduction, ``Consequently\ldots Furthermore.'' It is a bit distracting and unnecessary and detracts from the overall quality of the paper, which is excellent. Go through the manuscript and remove as much of this is possible.

\textcolor{blue}{Fixed. Change S16}

4. Line 16, remove ``, e.g.,''

\textcolor{blue}{Fixed. Change S2}

5. Line 25, ``\ldots small to intermediate size primary and secondary tumors.'' You use small and intermediate in the discussion. Not sure if you need primary and secondary here.

\textcolor{blue}{We decided to keep ``small secondary tumors'' as this is an important insight from our model.} 

6. Global search on ``missegregation'' to make sure you use it with/without hyphen consistently.

\textcolor{blue}{Fixed. Change S4}

7. Line 65, ``\dots like mutation\ldots''

\textcolor{blue}{Fixed. Change S5}

8. Lines 120/121, describe susceptible, resistant, and tolerant here.

\textcolor{blue}{We have aded a description of susceptible, resistant, and tolerant in the main text. Change S17}

9. Table at 142, tab the right column so that each column lines up from row to row.

\textcolor{blue}{Fixed. Change S6}

10. Line 221, put "Table 1" in parentheses in a previous sentence.

\textcolor{blue}{Fixed. Change S7}

11. Sometimes you set variable names between commas, sometimes in parenthesis, sometimes without a leading comma. I would recommend a consistent strategy.

\textcolor{blue}{We believe that both parenthesis and commas are appropriate depending on the context.}

12. Line 297, ``\ldots as it reduced the threshold tumor sized needed for rescue.''

\textcolor{blue}{Fixed. Change S8}

13. Lines 334-343: ``In the above\ldots However\ldots Moreover\ldots Hence\dots Therefore\ldots ''

\textcolor{blue}{We have removed all of the connecting words highlighted. Change S19}

14. Remove comma end of line 334

\textcolor{blue}{Fixed. Change S9}

15. Line 343, ". . .standing genetic variation. . ."

\textcolor{blue}{Fixed. Change S10}

16. Line 356, ". . .de novo genetic variation. . ."

\textcolor{blue}{The correct terminology is "standing genetic variation" as we have $\tilde{N}_a^*$ as the numerator in equation 6.}

17. Lines 384-386, offer a reason for the statement that starts "Of particular interest. . ."

\textcolor{blue}{We are interested in the case of tolerant aneuploidy given that this represents a two step evolutionary rescue scenario. Change S11}

18. Line 438, "Decreasing the sensitive cell division rate will also decrease the cancer recurrence time." Note: This is a perfect example of how the leading word "Additionally" is unnecessary bc you use "also" in the sentence.

\textcolor{blue}{Decreasing the sensitive cell division will increase the cancer recurrence time. Change S12}

19. There a couple of places in the Discussion where you use ``fast enough'' to communicate the idea of timing with regard to aneuploidy. ``Fast enough'' indicates something about rate (and maybe timing), but ``early enough'' is about timing. Consider whether you want to use one or both to get this idea across. (Lines 474 and 477).

\textcolor{blue}{We have changed both usages to make it clear that we are talking specifically about the rate of occurrence of aneuploidy, not the timing. They now read ``as long as it occurs frequently enough'' and ``if the rate of occurrence of aneuploidy is high enough''. Change S20}

20. Line 481, swap "the resistance mutation" with "a resistance mutation"

\textcolor{blue}{We have made this change. Change S13}

21. Line 499, ". . .aneuploidy provides a partial. . ."

\textcolor{blue}{Fixed. Change S14}

22. Line 523, ". . .primarily by de novo. . ."

\textcolor{blue}{Fixed. Change S15}

%%%%%%%%%%%%%%%%%%%%%%%%%%%%%%%%
\textbf{Reviewer $\#$2}

Thanks to the authors for improving the connection with previous rescue theory. A few minor comments on the responses to my comments below. Line numbers in tracked changes version.

\textcolor{blue}{We thank the reviewer for their patience and thorough suggestions. At this point, they have basically taught us the evolutionary rescue literature. 
This was going above and beyond their responsibilities. 
We apologize that we didn't do this ourselves, and really appreciate the reviewer's efforts.}

lines 69-93: The review of multi-step rescue in the intro makes some helpful connections but the logical flow could be improved. Yes, Martin et al 2013 extended Orr $\&$ Unckless 2008 by allowing for multiple different rescue mutations to occur during wildtype decline, but for the purposes of this paper the more important (and yet unmentioned) extension is to allow rescue by multiple (and in particular, two) mutational steps. I think it is then logical to skip to Osmond et al 2020 and instead of focusing on the fitness landscape, which is irrelevant here, focus on the fact that they approximate the 2-step result of Martin et al 2013 in two regimes: sufficiently subcritical and sufficiently critical single mutants. Ignoring the fitness landscape, this is essentially the same result as in Iwasa et al 2003 and 2004, who were specifically thinking about drug resistance and host shifts. I know I mentioned the 2-locus Uecker and Hermisson (2016) result, as it is 2-step rescue, but without more explanation its not clear how it is relevant to the current paper and, in its current placement, breaks up the flow.

\textcolor{blue}{ We have tried to make these changes. The relevant paragraphs now read:}
\begin{displayquote}
	\textcolor{blue}{
    There has been extensive theoretical analysis of evolutionary rescue. Orr and Unckless (2008,
    2014) thoroughly investigated the simplest case in which there is a specific single mutation or low-
    frequency variant that i ssufficient to rescue the population. 
    Martin et al. (2013) extended these results
    to consider the effects of a range of rescue genotypes that can vary in both growth rate and variance in
    offspring number. 
    Importantly for the present work, they also argued that two-step processes involving
    a ``transient'' intermediate genotype with growth rate close to zero may be an important form of rescue;
    however, they only calculated that the rate of such rescues should scale with the total number of such
    transient genotypes produced. Osmond et al. (2020) derived approximate expressions for the rate of
    these two-step rescues in limiting cases and found that for some parameter values two-step rescue is
    more likely than rescue by a single step.}
    
    \textcolor{blue}{
    Resistance to cancer therapy is a natural application of evolutionary rescue theory (Alexander et al, 2014). 
    Iwasa et al (2003) used multi-type branching process theory to approximate the probability that a population under strong selective pressure can survive extinction, focusing on the case in which the cancer must get two mutations to be rescued, with the single-mutant still dying rapidly.
    Iwasa et al (2004) used a similar approach to find simple approximate expressions for the probability that a single lineage would evolve to survive via a wide range of possible mutational pathways.
    These results are essentially the same as those described later in the general context by Osmond et al (2020) for the probability of evolutionary rescue by multiple mutations,
    which we mentioned in the previous paragraph\ldots}
\end{displayquote}



lines 126-127: I think the last sentence of the intro, "This is similar to part of Osmond et al. (2020) but focused on a different fitness landscape.", could cause a little confusion. In Osmond et al 2020 there is very explicitly a fitness landscape, with mutational kernels, such that the fitnesses of mutations that arise from a genotype depend on that genotype's fitness. In constrast, here there are three genotypes with independently chosen fitnesses, $\lambda_i - \mu_i$. To avoid confusion I would perhaps just attach a small statement to the previous sentence (about studying 2-step rescue), e.g., ", as in Osmond et al 2020 and Iwasa et al 2003, 2004." and avoid mention of landscapes.

\textcolor{blue}{We have made this change.}

lines 582-585: The same issue about a landscape arises here. Again, I don't think it is helpful to think of the current paper as Osmond et al 2020 with a different fitness lanscape. Relatedly, I find the sentence "Comparing our results to the evolutionary rescue literature, we complement the studies by Iwasa et al. (2003) and Osmond et al. (2020) on multi-step rescue by deriving simple approximations for a different fitness landscape" confusing, in part also because Iwasa et al 2003 say that their results hold for any fitness landscape. Why not just drop that sentence and move on to the 1 vs 2 step rescue result?

lines 585-586: "Similar to Osmond et al. (2020), we find that two-step rescues can be more likely than simple one-step rescues." I think it would be worth adding the reason, because of a higher mutation rate to stepping-stone vs rescue mutations.

\textcolor{blue}{We have made both these changes. The paragraph now begins, ``Similar to Osmond et al (2020), we find that two-step rescues can be more likely than simple one-step rescues, because the mutation rate to stepping-stone (aneuploid) genotypes can be much higher than that to fully resistant genotypes.''}

lines 606-607: "Future work could test when these effects have significant effect on the results of our model." This sentence could be replaced with something more informative. I.e., one could hypothesize that rates of aneuploid loss are small enough relative to the fitness gain over the wildtype to ignore and that allowing a resistant euploid to have a higher fitness than a resistant aneuploid would decrease the importance of 2-step rescue (by some relatively small constant proportion?).

\textcolor{blue}{Agreed. We have rewritten this paragraph as follows:
\begin{displayquote}
    Our model neglects the possibility of back-mutations from aneuploidy to euploidy, and also the possibility that the fitness of a euploid cell with a resistance mutation may be higher than that of an aneuploid cell with the same mutation. 
    Both of these would be expected to reduce the importance of two-step rescues via aneuploidy.
    But for the former, unless the rate of loss of extra chromosomes is extremely high (for estimates in yeast, see Hose et al (2024)), comparable in magnitude to the growth rate $r_a$, we expect that its effect on the dynamics will be negligible.
    The latter possibility, in which aneuploidy substantially reduces the mutant growth rate, seems more likely
    to have an effect. 
    Including it would be a straightforward extension to our model.
    In this case, it could actually be more important to consider the possibility of the loss of aneuploidy, 
    as one would need to check the relative rates of the simple \textit{sensitive} $\rightarrow$ \textit{euploid mutant} path and the three-step \textit{sensitive} $\rightarrow$ \textit{aneuploid} $\rightarrow$ \textit{aneuploid mutant} $\rightarrow$ \textit{euploid mutant} path (Kohanovski et al, 2024).
\end{displayquote}
}

lines 392-393: The equivalent result in Martin et al 2013 is also for 1-step rescue, not 2-step rescue. See their eqn 3.8, which is derived from eqns 3.3 (de novo 1-step) and 3.6 (standing variance 1-step).

\textcolor{blue}{Fixed.}

appendix D: Smith $\&$ Haigh 1974 should be Maynard Smith $\&$ Haigh 1974.

\textcolor{blue}{Fixed.}

%\textcolor{red}{Remus comments:}
%
%\textcolor{red}{We change "rapidly" to "easily" because $\tau_m<\tau_a$ conditioned on evolutionary rescue and the word "rapidly" suggests that aneuploid speeds up the process of rescue which is not true. Change RS1}
%
%\textcolor{red}{Recurrence time is a decreasing function of sensitive and aneuploid growth rates not an increasing function as stated in the manuscript. Change RS2}
%
%\textcolor{red}{"anuploids"$\rightarrow$"aneuploids". Change RS3}
%
%\textcolor{red}{"aneuplidies"$\rightarrow$"aneuploids". Change RS4}
%
%
%



\bibliographystyle{plain}
\bibliography{evo2022}
\end{document}