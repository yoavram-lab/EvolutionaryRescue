\documentclass[12pt]{extarticle}
\usepackage{geometry}
\geometry{
a4paper,
total={170mm,257mm},
left=20mm,
top=20mm,
headheight=12pt
}

\usepackage{xcolor}
\usepackage{amssymb,amsmath,amsthm}
\usepackage[comma,sort]{natbib}
\usepackage{commath}
\usepackage{url}
\usepackage{graphicx} % Use pdf, png, jpg, or eps§ with pdflatex; use eps in DVI mode
\usepackage[T1]{fontenc}

\renewcommand{\d}{{\rm d}}
\newcommand{\e}{{\rm e}}

% Title page
\title{
	Response to reviewers: \\ Evolutionary rescue by aneuploidy in tumors exposed to anti-cancer drugs \\ 
}


\begin{document}
\maketitle

%%%%%%%%%%%%%%%%%%%%%%%%%%%%%%%%
\textbf{Reviewer $\#$1}

%Reviewer Summary:
%In this paper, the authors consider the role of aneuploidy in creating drug-resistant tumor cell lines in cancer. Specifically, they address how aneuploidy contributes to evolutionary rescue of a tumor cell population following treatment with tumor-cell killing drugs. They explore how different types of aneuploidy (stationary, tolerant and partially resistant aneuploid cell populations), time, mutation, rates of change, growth rates of drug resistant cells, pre-treatment aneuploid cell populations, and standing genetic variation, all might contribute, and importantly be exacerbated by aneuploidy, leading to the persistence of tumor cells following chemotherapy that result in a recurrence of tumor, even well past the point of extended remission. The authors use evolutionary models to identify the probability that a tumor could survive and proliferate under different genetic scenarios, rooting their parameters within ranges defined by previous studies, and identify several ways in which aneuploidy may serve a role, directly or in stepping-stone fashion, to facilitate evolutionary rescue of tumor cell populations, especially in smaller tumors often characteristic of primary tumors following resection and secondary tumors following metastasis.

%Major Edits:

I have no major edits to recommend on this manuscript and I am comfortable with the authors considering the minor edits and making adjustments as they see fit.

Minor Edits:

1. You do an incredible job of walking the reader through this manuscript, making clear your definitions of important concepts, like evolutionary rescue, along the way. 

\textcolor{blue}{Thank you for your kind feedback! We are glad our explanations were clear and helpful.}

Two places where you didn't do that, but that could really push this paper over the top into perfection territory, are the following: 1) offering your reader an explanation, even if brief, of the multi-type branching method central to your analyses/models, and why it was the right selection for this endeavor among other methods or evolutionary modeling frameworks used in the past or available to you,

\textcolor{blue}{ % change 53 
We have added a comment in line 239: ``We assume independence between clonal lineages starting from an initial population of $N$ sensitive cells (we check the effect of density-dependent growth on our results below) and therefore, we use multi-type branching processes to model the dynamics of the cancer cell population. Multi-type branching process models the growth and evolution of populations with distinct types, capturing dynamics like mutation and selection. In evolutionary rescue, it predicts the probability of population survival by tracking adaptive mutations that emerge and spread under environmental stress.''
}

2) putting something real-world up against the parameter ranges you used to start the models, or just a brief statement about how they were established from the literature you drew upon (i.e., from experimental tests, patient data, other models, etc.). The latter is just in an effort to put something real to the numerical ranges that a reader can relate to before launching into the modeling.

\textcolor{blue}{ % change 54
We have updated the ``Parameterization'' section (line 159): ``We parametrize most of our simulations by considering melanoma cells and rely on Rew and Wilson (2000) and Bozic et al. (2013) for the division and death rates, respectively.  
Rew and Wilson (2000)  report \emph{in vivo} measurements of the potential doubling times (the waiting time for the number of cells in the tumor to double, disregarding cell death) for a large set of cancer types. The division rate is obtained as $\lambda=\log{2} / T \approx 0.1$ per day. We select this to be the division rate for sensitive and mutant cells. This is a reasonable assumption given that we assume that the mutation makes the cancer cells immune to the chemotherapeutic drugs. 
Bozic et al. (2013) report the growth rate $\Delta_s$ for sensitive melanoma cancer cells, from patient data, from which they deduce the death rate $0.11 \le \mu_s \le 0.17$. We use  $\mu_s=0.14$ per day. Additionally, they observed the growth rate of cancer cells before treatment to be 0.01, which we use as the growth rate of mutant cells, which are resistant to the drug. Thus, we use $\mu_m=0.1-0.01=0.09$ per day as the death rate for mutant cells.'' \\
We also have another paragraph in the same section about parameterization of the model using evidence from breast cancer.
} 

2. Do a search for multi-type branching and make sure you consistently use the hyphen or not.

\textcolor{blue}{%change 31
We have made the change to use the hyphen in all mentions of ``multi-type branching'' such that it is consistent with textbooks such as Athreya $\&$ Ney (1972).
} 

3. You identify that at least in certain scenarios aneuploidy plays an important role in the evolutionary rescue of tumor cell populations following chemotherapy, at least with one type of drug, and maybe limited to small primary or secondary tumors. But, we never really know the extent that this scenario is applicable. Is it specific to this particular chemotherapy, small tumors, only certain types of cancer? In conjunction with this, there are places in the manuscript where you say things like ``for most parameter values aneuploidy will either play no role in evolutionary rescue or will be the main driver of adaptation without requiring any mutation (one step).'' Both parts of the quoted sentence stand in opposition to one of the more interesting scenarios found in your modeling, which is that aneuploidy can buy a tumor cell population enough time to accumulate a mutation that leads to evolutionary rescue. Like in point 1, making clear the broader implications of your findings, or the limited scenarios in which they might be applicable, pushes this paper over the top into perfection territory because it roots your reader into something relatable, manageable, or applicable to patient care. If the answer is ``I don't know how applicable this finding is or what the broader implications might be,'' you could then add something to the future research section that talks about how this might be addressed experimentally or theoretically. It's ok to not know for sure, but you shouldn't pass the opportunity to talk about it in limits and future research.

\textcolor{blue}{
Thank you for this remark, you are correct to note that this statement is in opposition to some of our results. We have removed this statement and instead, we now write (line 523): ``The `stationary' scenario, in which the aneuploid growth rate is close to zero in the presence of drug, is particularly interesting. Although this may seem like a small region of parameter space in a general model of evolutionary rescue (Fig. 4A, inset), it could be biologically significant for some tumors under specific drug therapies. 
Our results suggest that identifying aneuploidies that are tolerant or stationary may be worthwhile, as these are also expected to enhance the probability of rescue (Figure 3) and extend the window of opportunity for rescue (Figure 6).''\\
We also added the following sentences to the Discussion.
In line 468, emphasis on new text): ``Our results show that the probability of evolutionary rescue increases with the initial tumor size $N$, the drug-sensitive growth rate $\Delta_s$, the mutation rate $v$, and the aneuploidy rate $u$. \emph{The latter indicates that aneuploidy, even when it only provides tolerance, increases the probability that the tumor will be rescued.}'' \\
And in line 493: ``As an example, we have parameterized our model using estimates for melanoma and triple-negative breast cancer (TNBC) cells under drug therapy from the literature. We find that in these cases we are in the third scenario, in which aneuploidy provides at least partial resistance to the drug. It remains to be seen which tumor type and drug combinations produce tolerant and stationary aneuploidies.''\\
The following sentences were already in the Discussion (line 476): ``In this scenario, aneuploidy provides two advantages. First, it delays the extinction of the population, providing more time for the appearance of the resistance mutation. 
Second, it increases the population size relative to a drug-sensitive population, providing more cells in which mutations can occur. 
Together, this increases the cumulative number of mutants that arise.'' and in line 481: ``We find that aneuploidy can significantly affect evolutionary rescue as it reduces the threshold tumor size by at least an order of magnitude even when aneuploidy only provides tolerance.''
} 

4. I would revisit some of your figures to make sure groups/categories are not colored in contradictory ways from one graph to the next (i.e., like ``no aneuploidy'' being one color on one graph and a different color in another graph or different categories (``no aneuploidy'' vs. ``aneuploid'' vs. ``stationary'') using the same color in different figures).

\textcolor{blue}{ %change 56
We have changed Figure 6A to be consistent with the other figures and changed the caption for Figure 6 to be consistent with the new coloring.
} 

5. The results on detection time associated with tumor size were not intuitive to me on first read. You might want to clarify or rewrite those sentences to make it more clear. It read to me like something about this finding/scenario changed the baseline detection size required to know the tumor cell population was still present and viable.

\textcolor{blue}{%change 48
We have rewritten the sentence about the mean detection time (line 453): ``We find that for small and intermediate-sized tumors the mean detection time is approximately equal to the mean recurrence time (i.e., $\tau_a^r\approx\tau_a^{M}$ for $N<N_m^*$).''
} 

6. Table 1 caption doesn't define what the asterisk means.

\textcolor{blue}{ %change 1
We have added the definition of asterisk in the table.
} 

7. You never tell your reader what ODEs is... won't be obvious to everyone. I would just right it out on page 9 where I think is the first time you use it.

\textcolor{blue}{%change 2
We have added an explanation of ODE (line 381): ``In this case the dynamics of the number of mutant cells is deterministic and can therefore be modeled by a system of ordinary differential equations (ODE), which describe how the number of mutant cells changes over time by its time derivative (eq.~D2).''
} 

8. Are the tests for standing genetic variation essentially genetic drift? Variation is present and in a tumor cell population wildly fluctuating in size due to treatment you just hit on a ``right'' variant stochastically when tumor size is small (less cells) that is resistant and it's driven to fixation? If so, using genetic drift in this section is rooting language and another way of describing alternative scenarios/hypotheses for tumor rebound.

\textcolor{blue}{In our model, where aneuploidy is deleterious prior to drug treatment, standing variation is generated mis-segragation and purged by selection. Since aneuploids are carry a considerable fitness cost, random drift is not expected to effect standing genetic variation. Standing variation actually increases with the population size, whereas the effect of drift decreases with the population size. We now explain this in line 130: ``We assume that $c$ is not too small (Table 1), so that standing variation is generated by a balance between the generation of anuploids and selection against them, without the effect of genetic drift.''
} 

9. Do we know if it's reasonable to assume the division rates between cell types (i.e., ``sensitive'' vs. ``mutant'') are the same? Maybe ground that assumption into something we know, or just say we are not sure if that is reasonable or not. This relates to point 1 and rooting the reader into these parameter values with something real or being honest about the assumption they are equal.

\textcolor{blue}{ % change 40
We now ground this assumption in results from the literature (line 163): ``The division rate is obtained as $\lambda=\log{2} / T \approx 0.1$ per day. We take this to be the division rate for sensitive and assume that mutant cells, which are resistant to therapy, have the same division rate. Indeed, doubling times for tumors after relapse is lower than that of primary tumors for a variety of tissues \citep{tezuka2007growth,rodgers2024glioblastoma} and metastatic melanoma has been shown to grow faster then primary melonoma \citep{carlson2003tumor}.''
} 

10. Mid page 7, ``decreases'' is mis-spelled 

\textcolor{blue}{%change 27
Fixed. 
} 

and I would swap out ``very well'' with something more descriptive where you use it.

\textcolor{blue}{
We changed ``the approximations perform very well'' to ``the approximations are accurate'' in two places.
} 
\\
\\
%%%%%%%%%%%%%%%%%%%%%%%%%%%%%%%%
\textbf{Reviewer $\#$2}

Summary

%This is a paper about the likelihood and time for a tumour to evolve drug resistance, via direct mutation or indirectly through aneuploidy. The authors develop a multi-type branching process model to approximate these quantities analytically and to simulate the full behaviour. They also parameterize their approximations and simulations for melanoma, with adaptive trisomies for chromosome 2 and 6. The approximations provide intuition and match simulations well while demonstrating, for instance, that even tolerant (ie, subcritical) aneuploids -- which appears to be the case for melanoma -- can substantially facilitate drug resistance (fig 3) and extend recurrence times (fig 6).

I think this is a very nice application of evolutionary rescue theory. The presentation is clear and the neat approximations provide intuition. Parameterizing for melanoma demonstrates the utility of the theory.

\textcolor{blue}{
Thank you, we are pleased you found the application clear and intuitive.
} 

My main concern (elaborated below), is that all of the theoretical results derived here have extremely close analogues in previous evolutionary rescue theory. This substantially reduces the novelty of this paper, in the sense that much of the intuition developed here has been given elsewhere. However, I do think it is useful to derive all of these approximations for a single model that can be parameterized with existing cancer data. What I would like to see, then, is the authors connect their results with existing rescue theory, which would allow them to clarify what is new here and also help develop a largely missing link between the evolutionary rescue and cancer literatures.

\textcolor{blue}{This is a great point. We have revised the Introduction accordingly.}

$\#$ Major comments

1. The paper is framed as a model about the evolution of drug resistance in tumours via direct mutation or indirectly through aneuploidy. However, the model is much more general than this, it is a model of 1- and 2-step evolutionary rescue that is then parameterized for cancer. Because of this, I would like to see the authors spend more time discussing what is already known about 1- and 2-step rescue, which should also elucidate why another model is needed rather than simply parameterizing a model that already exists. I do think the authors can justify a new model, but I would like to see this developed with a more insightful review of existing models (currently Intro mentions some models exist but doesn't explain why they aren't sufficient) and a comparison of results derived here to those derived previously (currently no rescue papers cited after Intro). In addition to clarifying what is novel here, these comparisons may help connect the currently disparate fields of evolutionary rescue and cancer (see fig 1 in Alexander et al 2014 Evol. Appl.).

In case it helps, my understanding of multi-step rescue theory is that there was an early model for the emergence of epidemics via multiple mutations (Antia et al 2003 Nature, and some follow ups like Yates et al 2006 PRSB and Alexander $\&$ Day 2010 Interface), a very general treatment of emergence and rescue via multiple mutations (Iwasa et al 2004 JTB), a 2-locus 2-allele rescue model (Uecker $\&$ Hermisson 2016 Genetics), and multi-step rescue models on a fitness landscape (Martin et al 2013, Osmond et al 2020), all of which also use branching processes. Many of the results in this paper have been derived in the above studies. For example, eqn 3 is essentially the 1-step rescue probability from Orr $\&$ Unckless 2008 (Am Nat, also their eqn 3) and eqn 4 is essentially the emergence probability from Iwasa et al 2004 (their eqn 3) multiplied by the number of first step mutations that occur during the wildtypes decline. Even more similarly, Eqn 4 is a special case of the results in Osmond et al (2020), who also used the reasoning from fitness valley crossing (Weissman et al 2009) to derive approximations for the probability of 2-step rescue, with the added complication of a distribution of mutation effects. Setting the distribution of fitness effects to a dirac and assuming fast wildtype decline reduces the first two lines of their eqn 8 to the first two lines of eqn 4 (the final line of eqn 4 being the well known case of 1-step rescue). The de novo vs standing variance result (eqn 6) was pointed out in Orr $\&$ Unckless 2008 (right side of page 163) and given in Martin et al 2013 (eqn 3.8). The rescue and recurrence times (eqns 8 and 9) are perhaps the most novel results derived here, but have some close similarities with times derived in Orr $\&$ Unckless 2014 (see eqn 18 and text S2 for rescue time and eqn 23 for recurrence time).

\textcolor{blue}{
We owe a serious debt of gratitude to the reviewer for this comment. They have basically done all the work for us. We have incorporated a review of the literature in the Introduction (lines 55-87), and refer to these papers throughout the Results at the equations mentioned. The only slight changes we have made are omitting the follow-ups to Antia et al 2003 and instead including Iwasa et al 2003 Proc B, which we really should have mentioned before and is arguably the closest thing in the literature to this manuscript.
Because the results mentioned by the reviewer are scattered across multiple papers, and because they all follow from simple branching
process approximations, we re-derive all results here for our model.
} 

2. I have two general modelling-choice questions whose answers may substantially influence the results. 1) How heritable is a particular aneuploidy?  2) Are aneuploids with a resistance mutation really equivalent to diploids with a resistance mutation? Can the modeling choices made be empirically justified, or can the model (or results) be generalized to include these complications?

\textcolor{blue}{% change y1
Aneuploidy is indeed heritable. The extra-chromosome is lost at a rate similar to it's gain. A recent paper estimated the per generation loss rate in yeast at 8e-5-6e-4, depending on the specific chromosome (Hose et al. Genetics 2024). However, this transition from aneuploidy to euploidy is rare enough that we can neglect it, because the euploid has lower fitness than the aneuploid in the presence of drug.
As for the fitness difference between euploid and aneuploid mutants, this is a good question and the answer probably varies depending on the drug, the chromosome, and other factors. We now address both your questions in the Discussion, in line 542: ``Furthermore, loss of extra-chromosome may occur at a high rate (for estimates in yeast, see Hose et al. (2024)) and the fitness of a euploid cell with a resistance mutation may be higher than that of an aneuploid cell with the same mutation. Future work could test when these effects have significant effect on the results of our model.''
} 

$\#$ Minor comments

- line numbers would have been helpful

\textcolor{blue}{We have added line numbers.} 

- abstract: first sentence could give more precise definition of evolutionary rescue

\textcolor{blue}{%change 32
We changed the first sentence in the abstract: ``Evolutionary rescue occurs when a population, facing a sudden environmental change that would otherwise lead to extinction, adapts through beneficial mutations, allowing it to recover and persist.''
} 

- abstract: 'fitter'$\rightarrow$'more fit'

\textcolor{blue}{Done} % change 3

- intro: 'the number of chromosomes and chromosome arms copies'$\rightarrow$'the number of chromosomes and chromosome arm copies'

\textcolor{blue}{Done.}% change 4

- intro: while there are some similarities, I do not think that evolutionary rescue is mathematically equivalent to the problem of crossing fitness valleys. If that was the case, you wouldn't have to create a new model here and could instead parameterize the model of Weissman et al 2009 for cancer. One difference, for example, is that with fitness valley crossing the population doesn't go extinct and so it is a question about when the valley will be crossed, while in rescue the wildtype is declining to extinction and so the question is largely if rescue will occur. I do agree that the case of two step rescue is similar to fitness valley crossing, but that is a much weaker statement.

\textcolor{blue}{
We removed the statement about the equivalence to valley crossing.
} 

- intro: there is a relatively arbitrary list of rescue papers given at the end of the paragraph, which could be replaced (or supplemented) with synthetic reviews (eg, Alexander et al 2014 Evol Appl, Bell 2017 Annual Reviews)

\textcolor{blue}{%change 45
We have added reference to the recommended synthetic reviews in line 61.
} 

- intro: 'there are three ways for a population to survive ...' - I think this statement would be better placed at the beginning of the paragraph, after which you focus on genetic adaptation

\textcolor{blue}{%change 28
We have  moved the statement at the beginning of the paragraph, line 60. 
} 

- intro: `most models focus on the probability that at most one mutation rescues the population' - I disagree with this statement. I think the focus is rather on the probability that at least one mutation rescues the population. That is the logic behind the $1-\exp(-Np)$ form: $\exp(-Np)$ is the probability that no mutations rescue the population. In addition to Wilson et al 2017, Osmond $\&$ Coop 2020 (Genetics) also calculate the probability rescue is soft and the expected number of rescuing mutations in a variety of rescue scenarios.

\textcolor{blue}{%change 41
We have removed the statement ``most models focus...''. Instead, we include a review of the evolutionary rescue literature (lines 55-87).
} 

- methods: 'The aneuploid division rate is selected...' - I found the grammar odd here.

\textcolor{blue}{%change 36
We repharased to (line line 175): ``For the aneuploid growth rate to be intermediate between those of sensitive and mutant cells, $\Delta_s\ll\Delta_a\ll\Delta_m$, we set the aneuploid division rate to be $0.06 \le \lambda_a \le 0.1$.''
} 

- methods: `We estimate c=...' - It sounds like you will estimate c from the observed fraction of aneuploid cells but that is not what you describe below.

\textcolor{blue}{%change 5
We revised the text to make it clear that we estimate $c$ from an experimental result and $f$ from $c$ (line 208): ``To estimate fitness cost of aneuploidy, $c$, we note that Lukow et al. (2021) mixed sensitive and aneuploid A375 melanoma cells at $1:1$ ratio, cultured them in a drug-free environment, and observed the ratio evolve as a function of time with the aneuploid cells declining to $15\%$ after 24 days. Thus, our estimate for the fitness cost is $c=\abs{\log\left(0.15/(1-0.15)\right)/24}\approx0.07$ per day (Chevin 2011).
We estimate the fraction of aneuploid cancer cells in the pre-treatment environment using the formula $f=\tilde{u}\lambda_s / c$, which produces an estimate of $f=10^{-3}\times10^{-1}/0.07$, that is, $0.14\%$ of pre-treatment cancer cells have the beneficial aneuploidy of interest.''
} 

- methods: is there a justification for not including density-dependence for the sensitives as well? This would hasten their initial decline, which would compound with the reduced initial establishment probability of the mutants. It seems odd they can exert competition on the mutants but don't themselves feel any competition.

\textcolor{blue}{We have now added density-dependence for the sensitives (Eq. at line 226). The results are unchanged.}

- methods: I got a 404 error for the source code link: https://github.com/yoavram-lab/EvolutionaryRescue

\textcolor{blue}{
The repository will become available when the paper is accepted. Meanwhile, the source code for our analysis can be privately accessed here: \url{https://www.dropbox.com/scl/fi/hrvqdocsr67qur7k59wxw/StanaEtAlCode.zip?rlkey=n7bcr3usx5ym6q3s2qkotanhx}
}

- results: there doesn't appear to be an eqn 1

\textcolor{blue}{Fixed.} %change 29

- results: eqn 3 is an alternate form of the classic rescue probability, eg, $1/N_m^*=u/r*2(s-r)$ in Orr $\&$ Unckless 2008. I think it would be helpful to mention this.

\textcolor{blue}{ % change 33
We now mention that eq. 2 (previously eq.3 ) is an alternative form of an eq. from Orr and Unckless 2008 (line 261).
} 

- results: eqn 4 is an approximation (fast wildtype decline) of a special case (dirac distribution of mutational effects) in Osmond et al 2020, who used the exact same reasoning from Weissman et al 2009, and is also very similar to eqn 3 in Iwasa et al 2004. I think it would be helpful to mention this.

\textcolor{blue}{%change 57
We now note the similarity of eq.~3 (previously eq.~3) to Osmond et al 2020 and Iwasa et al 2004 (line 269).
} 

- results: '$T^*$ decerases'$\rightarrow$'$T^*$ decreases'

\textcolor{blue}{Fixed.} % change 27

- results: I thought it was interesting that the second scenario in eqn 5 was the geometric mean of the other two scenarios. Is there some intution for why this is so?

\textcolor{blue}{We agree, but we couldn't come up with such intuition.} 

- results: I thought it might be helpful to give a little more intution about the effect of simultaneously increasing the aneuploid birth and death rates (paragraph starting below eqn 5). The reason why this leads to a decrease in $T^*$ and pushes the system towards the second scenario is that the mutation rate has increased, meaning a lineage doesn't have to persist for as long to almost surely lead to a successful rescue mutant, while the expected persistence time hasn't changed, making it more likely that a rare long-lived aneuploid lineage will lead to rescue.

\textcolor{blue}{%change 51
We have added the recommended comment (line 303): ``Interestingly, increasing both the aneuploid division rate, $\lambda_a$, and the aneuploid death rate, $\mu_a$, such that the growth rate $\Delta_a$ remains constant, leads to a decrease in $T^*$, pushing the system to the second scenario. This is because increasing $\lambda_a$ causes a decrease in $T^*$ as  it increases the effective mutation rate $v\lambda_a$ (as mutations mostly occur during division) and a lineage does not have to survive as long in order to generate a successful mutant.''
} 

- results: I don't get the logic at the end of this paragraph, that your results are consistent with aneuploidy conferring resistance by decreasing the division rate. My understanding is that your result says that, holding aneuploid growth rate constant, increasing aneuploid division rate helps until you reach scenario 2 (effectively stationary/critical aneuploids) but beyond that doesn't help. I don't see how reducing division rate helps rescue.

\textcolor{blue}{%change 52
We have removed the sentence about these results being consistent with experimental finding.
} 

- results: if $N_a^*$ is similar to the detection threshold, then you predict that any tumour detected is very likely (eg, $>50\%$) to evolve resistance? Does this fit with data on resistance rates?

\textcolor{blue}{ % change 58
We believe that our model does fit with data on resistance rates, and the text now reads (line 316): ``Interestingly, the threshold between small and intermediate tumors, $N_a^*$, is similar to the tumor detection threshold of $4.19 \times 10^6$ cells for a wide variety of tumors~\citep{avanzini2019cancer}. We note that vemurafenib-treated melanomas (i.e. melanomas with sizes above the detection threshold) have a probability $>50\%$ to relapse \citep{piejko2023long,handa2022long}.''
}

- results: if aneuploidy may increase the mutation rate and this is easy to implement in the model, why not do it from the start, and reduce to the simpler version $v_a=v$ when needed?

\textcolor{blue}{
We start with the simpler model (same mutation rate in aneuploid and sensitive cells) and include the more complex model (different mutation rates) in Appendix H as we think the presentation is already complex enough and the literature does not have good estimates for aneuploidy induced mutation rates in cancer cells.
} 

- results: 'cancer cells(' $\rightarrow$ 'cancer cells ('

\textcolor{blue}{Fixed.} % change 25

- results: eqn 6 has been pointed out by Orr $\&$ Unckless 2008 and Martin et al 2013

\textcolor{blue}{%change 42
We now mention that this equation (eq.~5 in the revision) has been previously shown by Orr and Unckless 2008 and by Martin et al 2013 in line 346.
} 

- results: much of the discussion of the results (eg below eqn 7) is about the probability of rescue, but the equations all give the inverse of this, critical tumour sizes. I find it a little annoying to keep converting between them, and presenting the probabilities of rescue would also help compare with previous papers. Either way is fine but perhaps something to think about.

\textcolor{blue}{
We agree that having both quantities can be confusing. But we think that in the cotext of cancer evolutionary rescue, the tumor size has a more intuitive interpretation than the rescue probability.
}

- results: I think the rescue times given in the main text (eqn 8) could be explained a little more. For small tumours, this is the 2-step version of the 1-step rescue time given in Orr $\&$ Unckless 2014 (the expectation of their eqn 18), $-1/r$. I.e., the faster the population declines the earlier rescue occurs (conditioned on it occurring), because that is when most mutations happen. For larger tumours this is just a deterministic constant population size result, where the successful mutation is produced at a constant rate.

\textcolor{blue}{%change 46
We have added the following text in line 393: ``For small tumors ($ N \ll N_a^*$), the mean rescue time is the two-step equivalent of the one-step result from Orr and Unckleass (2014; expectation of eq.~18). The mean rescue time (conditioned on rescue occuring) is a decreasing function of the sensitive and aneuploid growth rates and independent of the other model parameters, including tumor size (blue line in Figure S6). This is because if the population rapidly declines but is then rescued, than the resistance mutation must have appeared early; if the population slowly declines, than mutations can appear later and the mean time will be longer. In our focus parameter regime, we have $\Delta_s=-0.04$ and $\Delta_a=-10^{-4}$, such that the mean rescue time is mainly determined by the aneuploid growth rate, $\tau_a \approx 10^4$ days. 
\\
For large tumors ($N \gg N_m^*$), the dynamics are equivalent to a scenario where rescue mutations appear at a constant rate, and the mean rescue time is independent of the aneuploid parameters ($u$, $\lambda_a$, and $\Delta_a$). Increasing the per division mutation rate, $v$, leads to the faster appearance of a rescue mutations and hence reduced mean rescue time. Finally, increasing the tumor size leads to shorter mean rescue time, as more sensitive cells can mutate to become resistant.''
}

- results: `aneuploidy promotes evolutionary rescue after $1/\Delta_a$' $\rightarrow$ `aneuploidy promotes evolutionary rescue after $1/\Delta_s$'

\textcolor{blue}{Fixed.} % change 37

- results: in eqn 9 the variable $p_m$ is used, but it is not defined in the main text

\textcolor{blue}{%change 26
We have added the definition of $p_m$ in line 421: ``where $p_m$ is the probability that a lineage starting from a single mutant cell escapes stochastic extinction.''
} 

- results: the final term of the recurrence time for small tumours is equivalent to the return time of Orr $\&$ Unckless 2014 with standing genetic variation (eqn 16), but different from their return time with new mutation (eqn 23). Why?

\textcolor{blue}{
The return time with new mutation (eqn 23) from Orr $\&$ Unckless 2014 can be written as:
\begin{align*}
t_{return}&\sim\frac{1}{s-r}\log\left[\frac{2N_0s(s-r)}{r}\right]\\
&\sim \frac{1}{s-r}\log[2N_0(s-r)]+\frac{1}{s-r}\log\left(\frac{s}{r}\right)
\end{align*}
Additionally, when $|s-r|<r$ we have:
\begin{equation*}
\log s \approx \log r +\frac{s-r}{r}
\end{equation*}
which can be rewritten as:
\begin{align*}
\frac{1}{s-r}\log\left(\frac{s}{r}\right)\approx \frac{1}{r}
\end{align*}
As a result, the return time with new mutation can be written as:
\begin{equation*}
t_{return}\sim \frac{1}{r}+\frac{1}{s-r}\log[2N_0(s-r)]
\end{equation*}
which is the one-step equivalent to our recurrence time for small tumors.
} 

- results: you mention that drugs increasing the death rate of sensitive cells delays recurrence, but given that mutations happen during division, it might be worth mentioning that drugs that decrease the division rate of sensitive cells by the same amount delay recurrence even more (holding mutant division rate constant - which is maybe why you didn't mention this for your parameterization, but seems like a worthwhile general point?).

\textcolor{blue}{% change 49
The text now reads (line 435): ``Drugs that significantly increase the sensitive death rate, $\mu_s$, but do not affect the sensitive division rate, $\lambda_s$, delay cancer recurrence. Drugs that decrease the sensitive division rate delay cancer recurrence time even more, because they effectively decrease the mutation rate (assuming mutations occur during division).''
}

- results: why doesn't standing genetic variance affect the mean rescue time? This might have something to do with what I just mentioned (successful mutants tend to arise early) and more clues might be found in Orr $\&$ Unckless 2014, who also compare times with and without standing genetic variance.

\textcolor{blue}{Standing genetic variation does not have a significant impact on mean rescue time because the percentage of cells that have the beneficial aneuploidy, in our parameterization, is only $0.14\%$.} 

- discussion: I find the discussion of the two advantages of aneuploidy for rescue a little confusing. Yes, aneuploidy delays population extinction and yes it increases the population size at a given time, but the important fact is that it increases the cumulative number of mutants that are created, which is the expression you give at the end of the sentence. Perhaps it would help to insert an additional sentence before the expression, like ``Together, this increases the cumulative number of mutants that arise.''

\textcolor{blue}{%change 39
Thank you for this suggestions. We have revised the text (line 476): ``... aneuploidy provides two advantages. First, it delays the extinction of the population, providing more time for the appearance of the resistance mutation. Second, it increases the population size relative to a drug-sensitive population, providing more cells in which mutations can occur. Together, this increases the cumulative number of mutants that arise (i.e., $Nuv\lambda_s\lambda_a/\abs{\Delta_s\Delta_a}$).''
} 

- discussion: the last sentence mentions that some cancer therapies are designed to increase aneuploidy, which felt like a new and important piece of information for me. Some discussion of this in the intro could be helpful as motivation for this work.

\textcolor{blue}{%change x1
Line 52 in the Introduction reads: ``Furthermore, some proposed anti-cancer drugs elevate the missegregation rate to fight cancer cells (Lee et al. 2016), as an extremely high chromosome missegregation rate is incompatible with cell survival and proliferation.''
} 

- discussion: there are no citations of evolutionary rescue theory in the discussion, despite most of the results already existing or having very close parallels in the rescue literature. I think it would be nice to build more connection between rescue theory and cancer here.

\textcolor{blue}{We have added a paragraph in the Discussion (line 520) on the connections to Iwasa et al 2003 and Osmond et al 2020, which we think are the two closest parallels for the main results. 
For the other more detailed expressions, we have added the references in the Results as they come up.} 

- table 1: what are the asterisks in the name column?

\textcolor{blue}{%change 1
We have added the definition of asterisk in the table.
} 

- fig 2: when aneuploids are tolerant rescue can also happen indirectly (caption incorrect for panel B)

\textcolor{blue}{%change 24
We changed the caption to ``When aneuploid cells are tolerant ($\Delta_a<0$) evolutionary rescue can occur indirectly, but direct mutation is the most likely path for evolutionary rescue''.
} 

- fig 4: give eqn $\#$ of exact formula (this goes for all formulas used in all figures)

\textcolor{blue}{%change 43
We added equation references the formulas of $N_a^*$ and $N_m^*$.
} 

- fig 4: $\lambda_m$ value given twice

\textcolor{blue}{%change 18
We changed $\lambda_m$ to $\lambda_a$ in the first instance in the caption of Fig. 4 .
} 

- appendix A: `decline are significant'$\rightarrow$`decline is significant'

\textcolor{blue}{Fixed.}  % change 19

- appendix A: in eqn A10, the $p_a$ and $p_m$ terms in the denominator seem to disappear in the second line (they will disappear in the third line anyway, given small mutation, but I think they should technically be in the second line)

\textcolor{blue}{Fixed.} % change 20

- appendix A: and in the third line of eqn A10 the $p_m$ in the numerator disappears (again, this doesn't matter for the result, eqn A11, but is an error and is required for the approximation to get from A10 to A11)

\textcolor{blue}{Fixed.} % change 21

- appendix C: $w_t\rightarrow s_t$

\textcolor{blue}{Fixed.} % change 7

- appendix C: I think the approximation in eqn C1 makes more sense by taking the Taylor series of the exponential term with respect to $\Delta_st$, rather than with respect to $t$ alone. For instance, you write $t + O(t^2)$, but the higher order terms are only smaller than t when $t\ll1$.

\textcolor{blue}{We have changed $\mathcal{O}\left(t^2\right)$ to $\mathcal{O}\left(\Delta_st^2\right)$. } %change 22

- appendix C: grammar mistakes in sentence below eqn C3

\textcolor{blue}{We have corrected the grammatical mistakes. } %change 8

- appendix C: I think it would be helpful to define $N_m^*$ and $N_a^*$ again in the appendix before using them

\textcolor{blue}{We have added the definitions of $N_m^*$ and $N_a^*$ in Appendix C. } %change 38

- appendix C: I would like to hear the justification for the initial expression of $\tau_a$ (integral of exponential) when $N>N_m^*$

\textcolor{blue}{%change 23
The justification, which is now included in Appendix C, is the following: 
\begin{align*}
\tau_a&=\int_0^\infty tf(t)dt=\int_0^\infty t\frac{d}{dt}F(t)dt=-\int_0^\infty t\frac{d}{dt}S(t)dt\\
&=-\left[tS(t)\right]_0^\infty+\int_0^\infty S(t)dt\\
&=\int_0^\infty S(t)dt,
\end{align*}
where $f(t)$, $F(t)$ and $S(t)$ are the probability density function, cumulative distribution function and survival function.
In our case, $S(t)=\exp\left(-\int_0^t R_a(z)dz\right)$.
Additionally, for $N\gg N_m^*$ we have $1-\e^{-Np_s}\approx1$. 
We have corrected a mistake, we now write $\tau_a=\int_0^\infty \e^{-\int_0^\tau R_a(t)\d t}\d\tau$
instead of $\tau_a=\int_0^\infty \e^{-R_a(\tau)}\d\tau$.
}

- appendix C: similarly, I think it would be nice to hear some justification for the approximations in eqn C9, ie, can ignore direct rescue and rescue via aneuploidy very unlikely

\textcolor{blue}{
The justificaion, also included now in Appendix C, is in the following. 
We know that
\begin{align*}
\tau_a=\int_0^\infty t\left(v\lambda_ap_ma_t+v\lambda_sp_ms_t\right)\frac{\exp\left[-\frac{uv\lambda_s\lambda_aNp_m}{\Delta_s-\Delta_a}\left(\frac{\e^{\Delta_s t}-1}{\Delta_s}-\frac{\e^{\Delta_a t}-1}{\Delta_a}\right)-v\lambda_sNp_m\frac{\e^{\Delta_s t}-1}{\Delta_s}\right] }{1-\e^{-Np_s}}\d t.
\end{align*}
Additionally, if $N\ll N_a^*$ then the probability distribution of evolutionary rescue time is skewed toward small values (i.e. evolutionary rescue occurs conditioned on it happening). As a result, we have: 
\begin{align*}
\frac{\e^{\Delta_s t}-1}{\Delta_s}&=\frac{1+\Delta_s t+\mathcal{O}\left(\left(\Delta_st\right)^2\right)-1}{\Delta_s}=t+\mathcal{O}\left(\Delta_st^2\right)\\
\frac{\e^{\Delta_a t}-1}{\Delta_a}&=\frac{1+\Delta_a t+\mathcal{O}\left(\left(\Delta_at\right)^2\right)-1}{\Delta_a}=t+\mathcal{O}\left(\Delta_at^2\right)
\end{align*}
We observe that the first term in the exponential can be approximated to be zero and the second term as -$v\lambda_sNp_mt$ where $t\ll1$. As a result, we can approximate:
\begin{equation*}
\exp\left[-\frac{uv\lambda_s\lambda_aNp_m}{\Delta_s-\Delta_a}\left(\frac{\e^{\Delta_s t}-1}{\Delta_s}-\frac{\e^{\Delta_a t}-1}{\Delta_a}\right)-v\lambda_sNp_m\frac{\e^{\Delta_s t}-1}{\Delta_s}\right] \sim 1
\end{equation*}
Additionally, since $N\ll N_a^*\ll N_m^*$, evolutionary rescue is more likely to occur through the trajectory $sensitive \rightarrow aneuploid \rightarrow mutant$. As a result,
\begin{equation*}
\tau_a\sim\int_0^\infty \frac{tv\lambda_ap_ma_t}{1-\e^{-Np_s}}\d t.
\end{equation*}
}

- appendix D: the fact that lineages conditioned to survive drift reach N earlier is nicely discussed in Orr $\&$ Unckless 2014, who connect it back to classic work on selective sweeps by Maynard Smith $\&$ Haigh 1974

\textcolor{blue}{We have added citations for Orr and Unckless 2014 and Maynard Smith and Haigh 1974. } %change 35

- appendix D: `it accurate' $\rightarrow$ `it is accurate'

\textcolor{blue}{Fixed.} % change 9

- appendix F: `Figure Figure' $\rightarrow$ `Figure'

\textcolor{blue}{Fixed.} %change 10

- appendix F: `our approximation are' $\rightarrow$ `our approximations are'

\textcolor{blue}{Fixed.}% change 11

- appendix F: `From eq. (F2) we obtain' $\rightarrow$ `From eq. (F1) we obtain'

\textcolor{blue}{Fixed.}% change 12

- fig S2: `the leads'$\rightarrow$ `that leads', `is in agreement' $\rightarrow$ `are in agreement', $\lambda_m$ value assigned twice

\textcolor{blue}{Fixed.}% change 13

- fig S3: `number of mutation'$\rightarrow$ `number of mutations', $\lambda_m$ value assigned twice

\textcolor{blue}{Fixed.}% change 14

- fig S5: $\mu_a$ value assigned twice

\textcolor{blue}{Fixed.} % change 15

- fig S6: `the leads' $\rightarrow$ `that leads', `lines, right' $\rightarrow$ `lines,', `is in agreement' $\rightarrow$ `are in agreement', `lines represents' $\rightarrow$ `lines represent', $\lambda_m$ value assigned twice

\textcolor{blue}{Fixed.} % change 16

- fig S9: `the leads'$\rightarrow$ `that leads', $\lambda_m$ value assigned twice

\textcolor{blue}{Fixed.} % change 17
\\
\\
%%%%%%%%%%%%%%%%%%%%%%%%%%%%%%%%%%%%%%%%%%%%%
\textbf{Reviewer $\#3$}

%Summary

%Stana et al. study the effects of aneuploidy (mutations that give rise to atypical karyotypes) on the survival of a tumor- a population of malignant cells- in presence of an intervention, here, an anti-cancer drug. While the role of aneuploidy in development of drug-resistance has been acknowledged in the field of evolutionary oncology, the precise mechanism by which it incurs drug-resistance is not clear.

%This manuscript is about understanding the evolutionary role of aneuploidy in drug-resistance in cancer. The authors model tumor evolutionary dynamics under treatment as a multi-type branching process, where the types correspond to three genotypes: sensitive to therapy, mutant (resistant), and aneuploid. The key assumption is that aneuploid cells are less-sensitive than sensitive cells to therapy. They parameterized their model by the pre/post treatment cost of aneuploidy, mutation rates and birth/death rates for each genotype.

%Stana et al. derive equations for relevant evolutionary quantities under multiple parameter regimes. The evolutionary quantities include the probability of evolutionary rescue, and time to recurrence. They further investigate the effects of standing variation, and density-dependent growth. A central finding is that aneuploidy can directly or indirectly result in evolutionary rescue. In the latter, it delays tumor extinction and provides a window of opportunity for mutations to rescue the population.

Overall, the paper proposes a rigorous evolutionary approach to understand the effects of aneuploidy in development of drug-resistance in cancer. The authors verify the accuracy of their derivations via extensive stochastic simulations. The paper is well written and the ideas are presented in a cohesive and rational manner. The accompanying python implementation appears sufficient (see below) to reproduce the results. The problem is interesting and subject of active research in evolutionary oncology, specially development of drug-resistance subpopulations (PMID: 39223250) and new avenues of targeting aneuploid cells (PMID: 38992122).

\textcolor{blue}{Thank you for the positive review of our manuscript.}

Major points

- The importance of time-resolved measurements of clonal dynamics have been recognized and more such datasets are becoming available (e.g., PMID: 34163070, PMID: 38987605). The simulation studies in this manuscript use a parameter regime largely based on a melanoma cell-line (A375) and treatment with a specific BRAF-inhibitor (vemurafenib). The relevance of the paper would be improved if consistent results can be shown in other datasets, including the citations above.

\textcolor{blue}{Following the reviewer's suggestion, we have evaluated our model using parameters from breast cancer (PMID: 34163070), see Methods (line 180), Table 2, and Figure 3B, which also appear below.\\
We have aldo added the following text to the Discussion (line 493): ``As an example, we have parameterized our model using estimates for melanoma and triple-negative breast cancer (TNBC) cells under drug therapy from the literature. 
We find that in these cases we are in the third scenario, in which aneuploidy provides at least partial resistance to the drug. 
It remains to be seen which tumor type and drug combinations produce tolerant and stationary aneuploidies. % change B
We have also compared the probability of evolutionary rescue between melanoma and TNBC (black and color lines, respectively, in Figure 3B).
This comparison suggests that TNBC has a higher probability of relapsing compared to melanoma, because smaller tumors have higher resuce probability. Indeed, the probability of relapse in TNBC patients is higher than 50\% in the first 3-5 years after diagnosis \citep{taushanova2023synchronous}, whereas in melanoma patients it is approximately 10-20\% in the first 5 years \citep{wan2022prediction,von_schuckmann2019risk}.'' % change A
}% change A

Minor points

- An exciting finding of the paper is that aneuploidy can act as an evolutionary stepping stone (Figure 4). However, in the discussion, you point out that only in a small parameter regime this would be the case, and in most cases, aneuploidy either directly results in evolutionary rescue or does not play a role. If this were the case, wouldn't you expect that a majority of cancer cells in a tumor to either have clonal/truncal aneuploidies or have no clonal aneuploidies at all? Given the prevalence of aneuploidies in patient cohorts, how can we interpret the restricted parameter regime that allows it?

\textcolor{blue}{
We have removed the misleading sentence ``aneuploidy will either not play any role in evolutionary rescue or will be the main driver of adaptation'', which is incorrect. Instead, we write (line 526): ``Our results suggest that identifying aneuploidies that are tolerant or stationary may be worthwhile, as these are also expected to enhance the probability of rescue (Figure 3) and extend the window of opportunity for rescue (Figure 6).''\\
Furthermore, in our model, we are interested in a specific type of aneuploidy which increases the fitness of cancer cells in an environment affected by anti-cancer drugs. As a consequence, we catalogued all other forms of aneuploids as sensitive. These aneuploidies can appear in patient cohorts because, for example, they drive cancer itself or increase genome instability, which leads to mutations and aberrations that promote cancer.
} 

- Code Base
The python scripts used for simulation studies are provided in the supplementary package, however the github link is not accessible.
It would improve the reproducibility of the manuscript if the code to reproduce individual figures is provided.
Documenting the code as well as adding a short tutorial will improve the reusability of the scripts.

- In your model, mutations appear to only confer resistance, however, most mutations are deleterious. How easy is it to incorporate the effects of deleterious mutations in your model? This should be addressed in the discussion.

\textcolor{blue}{Our focus is on mutations that confer resistance. We assume deleterious mutations can occur with equal rates in all genotypes and therefore would not change the dynamics. We now explain these assumptions in line 215: ``Note that when we refer to drug-sensitive cells, we include those cells that have any aneuploidy that does not affect fitness in the presence of the drug. Additionally, we focus on mutations that confer resistance, neglecting deleterious mutations (which are more common). We assume deleterious mutations and other aneuplidies occur at similar rates across genotypes and therefore neglect their effect on the dynamics.''
} 

- You report the exact parameters used in most figures. Sometimes it is difficult to follow why a specific value for a parameter was chosen. It would improve the readability if you included a table where the variation in the parameter used is noted and explained.

\textcolor{blue}{The variation in our parameters in noted in Table 1 and the new Table 2 and is obtained from the papers which are cited there.} 
\\\\
%%%%%%%%%%%%%%%%%%%%%%%%%%%%%%%%%%%
\textbf{Associate Editor Comments}

The paper sits in an odd place, as noted by reviewer $\#2$. Existing theory already explores evolutionary rescue by one and two step mutations, and it is hard to see the connection between the current theory and the existing theory. Which of the results are new? It is important that the reader knows what is novel. If similar results are derived in a different way, please explain the connection. When the quantity considered is different (e.g., timing of certain events), please alert the reader and maybe explain why these quantities are particularly relevant in the context of cancer. Reviewer $\#2$ has many helpful suggests along these lines.

Another oddity is that the theory developed applies much more broadly to evolutionary rescue scenarios, but the presentation makes it seems that the results only apply to cancer. Similarly, the initial facilitating mutation could be any type of mutation, not just aneuploidy.

A sufficient revision must, in my view, properly connect this work to the existing theoretical literature on evolutionary rescue (as noted by reviewer $\#2$) and to the broader empirical literature on cancer about multi-step drug resistance (as noted by reviewer $\#3$).

\textcolor{blue}{
We agree with all of these points. We have revised the introduction and discussion accordingly, see lines 55-87, 520-529.
}

Minor comments
Introduction:
* 10 millions people die: add globally

\textcolor{blue}{We have made the recommended change. }  % change 30

* mathematically equivalent: this is incorrect because fitness-valley crossing papers typically ignore population size dynamics, whereas evolutionary rescue is explicitly set in the context of a declining population size.

\textcolor{blue}{We removed the statement about the mathematical equivalence of valley crossing and evolutionary rescue.}

* how multiple mutations contribute: several important papers on this subject are referenced later but not in this paragraph. Please revise, reviewing the existing theory and saying how the current theoretical work differs.

\textcolor{blue}{We have added the material, as described in our response to reviewer 2.}

Methods:
* The methods only describe the simulations, and the results focus on analytical results that don't follow from the methods. I suggest expanding the ``Methods'' to discuss the general approach, model structure, and quantities derived (you can rename this as ``Model'') and then move to ``Results''.

\textcolor{blue}{
The Methods section has several subsections: ``Evolutionary model'' that describes the model, from which the analytic results are derived; ``Stochastic simulations'', which describes the simulations used to validate the analytic results; ``Parameterization'', which describes how we parameterize the model; and ``Denstity-dependent growth'', which described the extended model that includes density dependence. Note that the ``Evolutionary model'' subsection is not focused on the simulations but rather described the general model that we are studying.} 

* Delta would be better as ``r'' for the growth rate

\textcolor{blue}{The manuscript has over 300 mentions of $\Delta$, plus mentions in the figures, and we do not want to introduce any errors by changing it.}
\textcolor{red}{I don't think this is a good argument. It's literally one command in \LaTeX to change all the ``$\Delta$''s in the text to ``$r$''s. So it would just be changing the figures, which isn't a big deal. I think we should do it.}

* The use of the terms ``tolerant'' (for negative growth) and ``resistant'' (for positive growth) is not standard (e.g., see different for fungi used by Berman $\&$ Krysan 2020 Nature Reviews Microbiology). Maybe alert the reader that ``tolerant'' is used in different ways in the literature and is used here solely for notational convenience to refer to mutations that, on their own, do not allow growth? Or replace with ``sub-critical'', a term already used in this context?

\textcolor{blue}{We have added a short explanation for ``tolerant'' and that it has other meanings in the literature. We also mention the equivalence to in ``sub/super/critical'' from Osmond et al 2020.
We use these terms because we are trying to translate concepts familiar to the evolutionary rescue community into ones more familiar to cancer researchers.} % change 47, 57

* ``antagonizing cell division'' - The meaning of this isn't clear. Rephrase as ``by extending the length of the cell cycle''?

\textcolor{blue}{We have made the recommended change from ``antagonizing cell division'' to ``by extending the length of the cell cycle''.}  % change 34

Results:
* Figure 7 - Add a numerical solution to equation (D3)? This should make it clearer that the non-monotonicity of the black curve is an artefact of the simplifying assumptions in (D4) and not a problem with the approach.

\textcolor{blue}{We have added the green line for eq. D3 to Figure 7 as suggested. It is very close to the blue line for eq. D4.}

* By the way, It wasn't clear to me what happened to the other large $\tau_a^r$ terms in the first portion of equation (D3) when approximating the equation into the form shown in (D4)? Could this be briefly clarified?

\textcolor{blue}{In the first term of (D3) we have: 
\begin{align*}
\frac{uv\lambda_a\lambda_s}{\Delta_s-\Delta_a}\left[\frac{\e^{\Delta_s\tau_a^r}-\e^{\Delta_m\tau_a^r}}{\Delta_s-\Delta_m}-\frac{\e^{\Delta_a\tau_a^r}-\e^{\Delta_m\tau_a^r}}{\Delta_a-\Delta_m}\right]\approx\frac{uv\lambda_a\lambda_s\e^{\Delta_m\tau_a^r}}{\left(\Delta_m-\Delta_s\right)\left(\Delta_m-\Delta_a\right)}
\end{align*}
which is significantly smaller then the second term of (D3):
\begin{align*}
v\lambda_s\frac{\e^{\Delta_s\tau_a^r}-\e^{\Delta_m\tau_a^r}}{\Delta_s-\Delta_m}\approx\frac{v\lambda_s\e^{\Delta_m\tau_a^r}}{\Delta_m-\Delta_s}
\end{align*}
} % change 44
\\
\textcolor{red}{I don't understand our answer here. Seems like we're assuming that tolerant aneuploids don't matter? And I don't understand what we're saying about ``large $\tau_a^r$''---large compared to what? Aren't we describing the regime of relatively small $\tau_a^r$ here? }
%%%%%%%%%%%%%%%%%%%%%%%%%%%%%%%%%%%%%%%%%%%%%

\textbf{Additional changes}


{\color{blue}


We have made the following additional changes:

$w_t\rightarrow s_t$. %change r1

$t\rightarrow \tau$. %change r2

The rates of the inhomogenous Poisson process are: %change r3
\begin{align*}
r_1\left(t\right)&=v\lambda_ap_ma_{t},\\ 
r_2\left(t\right)&=v\lambda_sp_ms_{t},
\end{align*}
and the expected number of successful mutants created by direct mutation and via aneuploidy until time $t$ is given by:
\begin{align*}
M_a\left(t\right)&=v\lambda_ap_m\int_0^ta_{z} d z = \frac{uv\lambda_s\lambda_aNp_m}{\Delta_s-\Delta_a}\left(\frac{\e^{\Delta_st}-1}{\Delta_s}-\frac{\e^{\Delta_at}-1}{\Delta_a}\right),\\ 
M_m\left(t\right)&=v\lambda_sp_m\int_0^ts_{z} d z = v\lambda_sNp_m\frac{\e^{\Delta_s t}-1}{\Delta_s}.
\end{align*}
 

Appendix C: all dummy variable used in integration are consistent. %change r4

Figure S6, S7, S8, S9: changed the subscripts of several variables from $w$ to $s$ . %change r5

``If a fraction $f$ of the cancer cells are aneuploid when the drug is administered then the rates at which the rescue mutations are generated can be written as'' $\rightarrow$
``If a fraction $f$ of the cancer cells are aneuploid when the drug is administered then the expected number of successful mutants generated until time $t$ is given by''. % change r6 

}

\bibliographystyle{agsm}
\bibliography{ms}
\end{document}